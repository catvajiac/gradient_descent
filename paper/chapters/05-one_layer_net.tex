\newpage
\section{Classifying Non-Linear Data}
The Softmax Classifier is good at classifying linearly separable data.
However, many classification problems involve data that isn't linearly
separable. In order to classify nonlinear data, we must add a nonlinearity to
our neural network architecture. There are many different ways in which this
can be done, but the current most popular function used to provide
nonlinearities in neural networks is called the REctified Linear Unit (RELU)
function, and is defined as follows:

\[ relu(x) = \begin{cases} 
      0 & x\leq 0 \\
      x & x > 0 
   \end{cases}
\]

By propagating our data through RELU on top of the softmax classifier, we will
create a one-layer neural network, which has the following different
parameters:
\begin{enumerate}
\item parameters $W^{(1)}$ and $b^{(1)}$
\item parameters $W^{(2)}$ and $b^{(2)}$ \item a non-linear function (the
$relu$ function in our case), denoted $\phi$, which maps $\mathbb{R}
\longrightarrow \mathbb{R}$.
\end{enumerate}

Our neural network now has more parameters, and so the prediction step is a
little more complicated. Data gets mapped to predictions by the following
process:
\begin{enumerate}
\item $A = XW^{(1)} + b^{(1)}$,
\item $Z = \phi(A)$,
\item $P = ZW^{(2)} + b^{(2)}$.
\end{enumerate}

Once again, we must use a loss function to optimize our parameters. We will use
the Softmax Loss function as we did before. (Note: current research shows that
other loss functions produce better results, but those functions tend to be
more complicated and are more computationally expensive to train. Everything is
a tradeoff).

We will need $\phi'$ to calculate our update formulas, but we note that
$\phi$ is not differentable at $x=0$. In place of $\phi'(x)$, we will use
$1(x > 0)$, where $1(C > D)$ evaluates to 1 if $C > D$ and 0 otherwise.

\subsection{Deriving $\nabla W^{(2)}$ and $\nabla b^{(2)}$}
Since the output layer, or last layer of our new architecture is identical to a
Softmax classifier, just with $Z$ in place of $X$, we compute $\nabla
W^{(2)}$ and $\nabla b^{(2)}$ to be

$$ \nabla W^{(2)}  = Z^T P, $$
$$ \nabla b^{(2)}  = \sum_{i=1}^m P_i. $$

% rewrite, X can be replaced iwth Z?
Deriving $\nabla Z$ would be done in exactly the same way as $\nabla X$. If
we define $\nabla Z_{uv} = \frac{\partial L}{\partial Z_{uv}}$, we now have

$$ \nabla Z = \nabla P W^{(2)}. $$

\subsection{Deriving $\nabla A$, $\nabla W^{(1)}$, $\nabla b^{(1)}$}
Recall that $Z = \phi(A).$

In order to calculate $\nabla W^{(1)}$ and $\nabla b^{(1)}$, we will first
calculate $\nabla A$. To do this, recall that $Z_{uv} = \phi(A_{uv})$
and that $L_i$ is a function of $Z_{uv}$ only if $i == u$ for any $i, u, \in
[0, m]$. We see that

\begin{align*}
  \frac{\partial L}{\partial A_{uv}}
  &= \sum_{i=1}^m \frac{\partial L_i}{\partial A_{uv}}\\
  &= \sum_{i=1}^m \frac{\partial L_i}{\partial Z_{uv}} \frac{\partial
  Z_{uv}}{\partial A_{uv}}\\
  &= \frac{\partial L_u}{Z_{uv}} 1(A_{uv} > 0)\\
  &= \frac{\partial L}{Z_{uv}} 1(A_{uv} > 0)\\
  &= \nabla Z_{uv} 1(A_{uv} > 0).
\end{align*}

So we have
$$ \nabla A = \nabla Z \bullet (A > 0), $$

where $a \bullet b$ indicates the element-wise multiplication of matricies $a$ and $b$.

We can define matricies $\nabla W^{(1)}$ and $\nabla b^{(1)}$ as we did with $W^{(2)}$ and
$W^{(2)}$ earlier. Since these matricies have the same relationship with $A$ as
$W$ and $b$ did with $P$ in the softmax classifier, we get
$$ \nabla W^{(1)} = X \nabla A, $$
$$ \nabla b^{(1)} = \sum_{i=1}^m \nabla A_i,$$

so $\nabla b$ is a vector containing the column-wise sums of $A$. 

\subsection{Update formulas}
Now that we have calculated the gradients for all of our parameters, we can
define their update formulas as follows:

\begin{align*}
W^{(1)} &= W^{(1)} - \eta \nabla W^{(1)},\\
b^{(1)} &= b^{(1)} - \eta \nabla b^{(1)},\\
W^{(2)} &= W^{(2)} - \eta \nabla W^{(2)},\\
b^{(2)} &= b^{(2)} - \eta \nabla b^{(2)}.
\end{align*}

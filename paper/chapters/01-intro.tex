\newpage
\section{Introduction}

The goal of image classification is to determine the main contents of an image,
commonly referred to as the image's \textit{class}. This is a more complex
problem than one may think when first studying it due to the variation in
images. While there are many algorithms that attempt to classify images, neural
networks are known to be most successful in doing so.

There are many different cases in which one can have an unideal image,
including viewpoint variation, scale variation, deformation, occlusion,
illumination conditions, and intra-class variation. These will be described in
detail later on in this paper, but they all describe ways in which images of
the same class can vary. For this reason, a complex algorithm is required to
classify images.

Neural networks classify images using a data-driven approach by processing many
examples from different classes of images. They have neurons, or units, that
learn the features of a dataset. Those units then propagate information to
other units. This process allows the algorithm to learn from observational
data.

Neural networks have two phases; training and inference. In the training phase,
they learn from a large dataset, and in inference, they apply that knowledge to
new testing data that hasn't been seen before. There's a rich mathematical
theory involved in designing a neural network, which will be described in
detail in Chapter \ref{ch:gradient_descent}.

Despite advances in hardware that have made it feasable to use neural networks,
they are still extremely computationally expensive to train, sometimes taking
weeks or months to do so. Current research is being conducted in order to find
ways to speed up neural networks by innovative technical companies, such as
Google, Facebook, and AMD. While it is not the focus of this paper to discuss
this topic, it's important to realize that new techniques are still being
developed as this is a growing field.

In Sections \ref{ch:image_classification} and \ref{ch:neural_networks}, we will
discuss background for image classification and neural networks. In Section
\ref{ch:gradient_descent}, gradient descent, the iterative algorithm that
trains a neural network will be discussed. In Section \ref{ch:python}, we will
discuss the implementation of a fully-connected network in Python.
